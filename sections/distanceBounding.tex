%Section on Distance bounding protocols

\subsection{Introduction}
	These protocols were designed to ensure an upper bound for a distance between to entities.
	They were introduced by \citeauthor{distanceBoundingProtocols}.
	Distance bounding protocols were evaluated by \citeauthor{secPos}.
	They discribe a protocol that implements this upper bounding for localization in wireless networks.
	The authors emphasize that their proposed protocol is very well suited to determine the distance,
	as it is quite accurate and that it is very secure.
	Although distance bounding means,
	that it can only ensure a certain minimum distance between the devices.
	An attacker could give fool the devices that they are more far away from each other.
	
\subsection{Method}
	The distance bounding protocol as described by \citeauthor{secPos}
	bases on the time-of-flight for electromagnetic waves which travel at the speed of light.
	The devices implementing this distance bounding protcols are required to operate very fast.
	Otherwise the postion information will not be very accurate.
	As the speed of light is a curcial constant,
	if the device takes only one nanosecond longer,
	the distance will be uncertain by about 30 cm.
	Hardware that allows this kind of processing speed is available for the Ultra Wide Band.
	The range of these devices and their physical properties,
	like weight and size would be suitable for PKE Systems.

	The protcol could implement a PKE system in the following manner:
	The vehicle permanently sends out a signal,
	which might be recieved by the CID.
	The signal does not need to be secure in any way,
	it could even be relayed by an attacker.
	This signal is just used to initiate the protocol.
	The CID will the start by sending a random nonce to the vehicle
	and there by initiate the distance bounding protocol.
	The vehicle will the respond with a challenge nonce,
	that is send to the CID in reversed bit order,
	starting with the most significant bit.
	As soon as the last bit is send a precise timer is started.
	The CID then XORs its initiating nonce with the challenge nonce send by the vehicle.
	The result will the be send in to the vehicle,
	starting with the lowest significant bit.
	When the XORed nonce is received the timer is stopped,
	and the time is converted to a distance.
	In a final step an authentication can be acquired using values commited from the CID nonce
	and a shared key.

% distance bounding as described in "secure positioning``/
