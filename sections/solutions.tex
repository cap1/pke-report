%Solutions Section for the pke-report

\subsection{Shortterm}
% shutdown
	As described in Section \ref{sec:relayAttacks}, relay attacks are serious issue.
	Especially because of its implications on overall vehicle security as described in Section \ref{sec:attackImplications}.
	Usually it is possible to disable the PKE system in the cars.
	Access to the vehicle can be gained by a backup key system.
	The CID usually encorporates a backup key so that disabling the PKE feature is possible.
	This of course lowers the ease of access.

\subsection{Midterm \& Longterm}
%improvement described in "Some attacks against vechicles pkes .. ."
	In order to make attacks more difficult the manufactures could hard their systems.
	\citeauthor{someAttacksPKES} elaborate som methods to harden PKE systems.
	They describe a Systems using communication in RF.
	This kind of communication would introduce a feedback loop into the attackers relay systems.
	The feedback loop could be easily detected by the PKE system so that access is prohibited.
	Although the the authors also present an attack with \textsl{three thieves}
	as described in Section \ref{par:threeThieves}, that could still
	carry out the attack.
	To prevent this kind of attack,
	\citeauthor{someAttacksPKES} propose a system with different signal strengths,
	transmitted by the CID.
	The vehicle can then check for the difference of the powerlevels of the incomming signal.
	To deploy the \textsl{three thieves} attack,
	the attackers have to approach the car from opposing directions to prevent a feedback loop.
	By sending out a signal with different powerlevels,
	it is much harder for the thieves to position themselves in the correct range.
	Also the difference of the powerlevels could be altered,
	if amplification, downconversion and upconversion are not implemented exactly.

	

% distance bounding as described in "secure positioning``/
