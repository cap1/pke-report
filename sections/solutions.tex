%Solutions Section for the pke-report

\subsection{Shortterm}
% shutdown
	As described in Section \ref{sec:relayAttacks}, relay attacks are serious issue.
	Especially because of its implications on overall vehicle security as described in Section \ref{sec:attackImplications}.
	Usually it is possible to disable the PKE system in the cars.
	Access to the vehicle can be gained by a backup key system.
	The CID usually incorporates a backup key so that disabling the PKE feature is possible.
	This of course lowers the ease of access.

\subsection{Midterm \& Longterm}
%improvement described in "Some attacks against vechicles pkes .. ."
	In order to make attacks more difficult the manufactures could hard their systems.
	\citeauthor{someAttacksPKES} elaborate some methods to harden PKE systems.
	They describe a Systems using communication in Radio Frequency (RF).
	This kind of communication would introduce a feedback loop into the attackers relay systems.
	The feedback loop could be easily detected by the PKE system so that access is prohibited.
	Although the the authors also present an attack with \textsl{three thieves}
	as described in Section \ref{par:threeThieves}, that could still
	carry out the attack.
	To prevent this kind of attack,
	\citeauthor{someAttacksPKES} propose a system with different signal strengths,
	transmitted by the CID.
	The vehicle can then check for the difference of the power levels of the incoming signal.
	To deploy the \textsl{three thieves} attack,
	the attackers have to approach the car from opposing directions to prevent a feedback loop.
	By sending out a signal with different powerlevels,
	it is much harder for the thieves to position themselves in the correct range.
	Also the difference of the power levels could be altered,
	if amplification, down-conversion and up-conversion are not implemented exactly.

%GPS for localization
	Other methods could be applied to ensure the correct localization of the key.
	Modern cars that incorporate a PKE System also include a satellite navigation system,
	which usually use the Global Positioning System (GPS) to determine their location.
	The PKE could make use of the positioning information gained from the GPS system.
	Although this requires the CID to also be able to determine its location by GPS.
	If the CID and the vehicle can communicate in a secure manner,
	this could be used for localization.
	Although battery lifetime and size of the CID would reduce the practicability of this system
	and the the system might not work in underground parking lots.
	The CID might be integrated in a smartphone,
	which usually incorporate a GPS device.
	Systems based on GPS could also be attacked,
	as pointed out by \cite{someAttacksPKES},
	but they are more difficult to carry out and are more expensive than simple relay attacks.
%Distance bounding
	The implementation of distance bounding protocols as described by \cite{secPos},
	might be the most secure approach to harden PKE systems against relay attacks.
	It allows the definition of a very clear upper bound for the distance between
	the CID and the vehicle.
	As described in Section \ref{sec:DBimp},
	the attacker can only fool the protocol into believing that he is farer away from the
	vehicle than he actually is.
	This is obviously not helpful fur his attack.
	Although it seems difficult for the manufacturers to implement those systems,
	as distance bounding protocols have been introduced more than 15 years ago  \cite{distanceBoundingProtocols}.
	The reasons might be that the necessary devices could not be build in an acceptable
	size or with a battery time that is long enough.
	