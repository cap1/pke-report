%Introduction part in the article on PKE
The access to cars has been managed by keys for a long time. %source or date would be nice...
Usually they enforce restriction to access into the vehicle
and igniting the engine.
Mechanical keys have evolved to be more difficult to forge.
Although the amount of electronics in these key-systems has steadily increased over time.
This enhanced security of cars and the ease of use,
but also introduced security problems.

\subsection*{Evolution of car keys}
The Customer Identification Devices hast changed a lot in car history.
Starting from physical keys,
the access can be enforced now by electronics only.

In the late 1990s, immobilisers have been introduced into car security.
In addition to a mechanical key, 
they incorporate an electronic device in the keys fob.
This device communicates with the cars electronics,
to prevent the "hot-wiring" of the car,
after physical access to the cabin has been achieved.
The strength of the immobilisers was increased with more sophisticated cryptography.
Early models used simple checks against values stored in the cars electronic control unit (ECU).
The strength of immobilisers was increased by incorporating more
sophisticated cryptography,
that can withstand replay-attacks.	%explain replay?

Remote keyless entry systems feature a key that can send out a signal,
which opens the car if its sensor is in range.
These system usually only unlock the vehicles doors,
and the authorization to start the engine is sill enforced by a physical key
which may incorporate an immobiliser.
These remote keyless entry systems work similar to immobiliser with a radio interface.
Likewise cryptography has to be applied to prevent replay attacks.

Passive keyless entry (PKE) system omit the button.
They listen on a certain frequency for a message from the vehicle
and reply if the message fits to initiate an authentication.
When the authentication succeeds, the vehicle will unlock its doors.
Now the Customer does not have to press any buttons,
and will be able to enter the car without interaction 
if the identification device is close enough.
The customer identification device usually integrates a backup key,
to allow access when the battery is drained.

This enhances the comfort but also introduces some security relevant aspects.
The vehicles security systems now have to be hardened
to withstand attacks on the localization of the key.
As the proximity of the key defines when the car is accessible and when not. 

\subsection*{Relayattacks}
These security problems introduced by PKE base on the problem of proper
localization in wireless networks.
Although it seems intuitive that the position of devices can be calculated by
triangulation, there are some attacks that allow the alteration of the position.

%picture triangulation

In wireless security these attacks are know as ``Wormhole-Attack''.
They use the same underlying mechanisms, as the attacks on PKE Systems.

%from wormhole attacks to relay attacks
