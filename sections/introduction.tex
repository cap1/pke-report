%Introduction part in the article on PKE
The access to cars has been managed by keys for a long time. %source or date would be nice...
Usually they enforce restriction to access into the vehicle
and igniting the engine.
Mechanical keys have evolved to be more difficult to forge.
Although the amount of electronics in these key-systems has steadily increased over time.
This enhanced security of cars and the ease of use,
but also introduced security problems.

\subsection*{Evolution of car keysystems}
	The Customer Identification Devices hast changed a lot in car history.
	Starting from physical keys,
	the access can be enforced now by electronics only.

	In the late 1990's, immobilisers have been introduced into car security.
	In addition to a mechanical key, 
	they incorporate an electronic device in the keys fob.
	This device communicates with the cars electronics,
	to prevent the "hot-wiring" of the car,
	after physical access to the cabin has been achieved.
	The strength of the immobilisers was increased with more sophisticated cryptography.
	Early models used simple checks against values stored in the cars electronic control unit (ECU).
	The capabilities of immobilisers was increased by incorporating more
	sophisticated cryptography.

	Remote keyless entry systems feature a key that can send out a signal,
	which opens the car if its sensor is in range.
	These system usually only unlock the vehicles doors,
	and the authorization to start the engine is sill enforced by a physical key
	which may incorporate an immobiliser.
	These remote keyless entry systems work similar to immobiliser with a radio interface.
	Likewise cryptography has to be applied to prevent replay attacks.
	Otherwise replay attacks are easy to carry out,
	as the attacker just can collect the signal send out by the owner
	and replay it near the car to gain access.

	Passive keyless entry (PKE) system omit the necessity to push a button.
	The key listens on a certain frequency for a message from the vehicle
	and reply if the message fits to initiate an authentication.
	Other systems the car only emit a signal if the door handle is pulled.
	When the authentication succeeds, the vehicle will unlock its doors.
	Now the Customer does not have to press any buttons,
	and will be able to enter the car without interaction 
	if the identification device is close enough.
	In side the car there are different signals send.
	These are lower in range so that the key has to be inside the car.
	Without having to put the key in a certain spot inside the vehicle,
	the engine will start with the push of a button.
	Working PKE system do not require any direct physical interaction between
	a key and lock.
	The customer identification device (CID) usually integrates a backup key,
	to allow access when the battery is drained. 

	This enhances the comfort but also introduces some security relevant aspects.
	The vehicles security systems now have to be hardened
	to withstand attacks on the localization of the key.
	As the proximity of the key defines when the car is accessible and when not. 

\subsection*{Relayattacks}
	These security problems introduced by PKE base on the problem of proper
	localization in wireless networks.
	Although it seems intuitive that the position of devices can be calculated by
	triangulation, there are some attacks that allow the alteration of the position.

	%picture triangulation
	Triangulation works by measuring the time it takes a signal to propagate
	to different antennas.
	Given a sufficient amount of antennas,
	it is possible to determine where the signal was emitted,
	by finding the point where the distance-radius cross. 				%TODO write it better!
	This method can be used,
	for example to track cellphones.	%TODO Zeit.de VDS Daten
	Triangulation is widely used but one has to keep in mind,
	that it solely determines the position where the signal was emitted.
	This position must not be the position where the signal has been generated in the first place.
	It is important to remember that it is quite easy to catch
	a certain signal and relay it to some other place.

	%from wormhole attacks to relay attacks
	Relay attacks work by this scheme.
	The Attacker has two devices,
	that are somehow connected.
	A receiving device allows the attacker to catch the signal where it is generated.
	The receiving device just captures the signal and
	without any alteration the signal is send to the second device.
	This device is used to emit the signal again,
	at the location desired by the attacker.
	Now the attacker can position the signal where want to.

	In wireless security these attacks are know as \textsl{Wormhole-Attack}.
	Some attacks are described by \citeauthor{wormhole}.	%TODO give example
	They use the same underlying mechanisms, as the attacks on PKE Systems
	and also base on the problem of proper localization in wireless networks.

	There have been some proposed solutions to prevent relay attcks and
	ensure a secure positioning in wireless networks.
	These \textsl{distance-bounding protocols} allow to determine a upper
	bound the distance between a two devices.
