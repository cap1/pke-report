%Introduction part in the article on PKE
Passive keyless entry Systems can be installed in a wide range of keys,
introducing some security problems that were not present before.
They differ from Remote keyless entry system by the fact that no
button has to be pressed to open the vehicle.
Remote keyless entry systems feature a key that can send out a signal,
which opens the car if its sensor is in range.
The Customer identification device is usually integrated in to the physical key
for the vehicle.

Passive keyless entry (PKE) system omit the button.
They listen on a certain frequence for a message from the vehicle
and reply if the message fits to initiate an authentication.
When the authentication succeeds, the vehilce will unlock its doors.
Now the Customer does not have to press any buttons,
and will be able to enter the car without interaction 
if the identification device is close enough.

This enhances the comfort but also introduces some security relevant aspects.
The vehicles security systems now have to be hardened
to withstand attacks on the localization of the key.
As the proximity of the key defines when the car is accessible and when not. 

These securityproblems introduced by PKE base on the problem of proper
localization in wireless networks.
Although it seems intuitive that the postion of devices can be calculated by
triangulation, there are some attacks that allow the alteration of the postion.

%picture triangulation

In wireless security these attacks are know as ``Wormhole-Attack''.
They use the same underlying mechanisms, as the attacks on PKE Systems.


