%%Relayattack part in the article on PKE

\subsection{Relayattacks}
\subsubsection*{Idea}
%Describe attack idea
	As described before relay attacks base on the problem of localization in
	wireless networks.
	Attacking a passive Key-less entry system,
	the attacker relays the probing signal emitted from the car to the victims
	customer identification device (CID).
	The CID the emits a signal that opens the doors of the vehicle.
	Now the attacker can enter the car and again relay the signals emitted by the
	car so that he can ignite the engine.
	
	This method has been practically tested on different PKE systems by %TODO \citea{} %DER Artikel
	The authors showed that it is an eqsy and feasible attack.
	It also does not require very specialized hard- or software and
	the required components can be easily aquired with a relatively small budget.
	The relay attack is transarparent to any higher layer cryptography,	%transport layer?
	and so can easily circumvent the security systems of most PKE systems.

\subsubsection*{Variants}
	There are some variants in attacking PKE systems.
	Altough the attack could be in principle be carried out by one attacker alone,
	the literature usually asumes at leats two attackers.

%two thieves
	Having two attackers performing a relay attack,
	the roles are split.
	The first thievs will try to get close to the victim and place
	the sending device close to the victims CID.
	This could take place in a store,
	where it doesnt raise attention to the victim if the attacker
	is in range of the CID.
	The second thieve will be near the car,
	probably parked on the stores own carlot,
	to capture its probing signals.

	Now the signals are relayed between the CID and the car,
	and the car might open and start,
	allowing the second thieve to drive away.

%three thieves
\subsubsection*{Material \& Methods}
%describe what materials where
%describe methods hwo to relate the signals, cable and wireless and its implications

\subsubsection*{Results}
%results francillion et al. achieved with real world cars

\subsection*{Implications}
\label{sec:attackImplications}
Gaining access to a vehicle as great implications for the overall security of a car.
Stealing the car is the straightforward approach,
but in combination with attacks on the cars internal electronic system,
malvolent attacks on the vehicle and its passenger are possible.

Modern vehicles are controlled by many small electronic computing units (ECU),
whom are interconnected over a bus system.
Prominent ECUs manage the engine, the breaks or the Electronic devices on the dasboard.
The more security relevant ECUs are connected via a priviledged,
high-speed bus,
where less important ones use a seperate lower-speed bus.
These bus-systems should be strictly seperate as specified in the standard. %TODO cite standard for OMB bla
Seperating the communication of the brakes and windscreenwipers,
shall ensure that the integral systems keep working,
when it is acceptable that less important ones can fail.

However the article form %TODO CITE!
shows that these security features and the standards were not implemented as intended.
%TODO \citea{}
point out what can be done if access to the to the electronic bus system can be achieved.
They were able to cut the engine and the engage single brakes,
even while driving.
%TODO \citea{}
point out that it is not possible to traceback certain attacks.	%verify that

%what harm can occur from gaining unnotices acces to the vehicle
