%%Relayattack part in the article on PKE

\subsection{Relayattacks}
\subsubsection*{Idea}
%Describe attack idea
	As described before relay attacks base on the problem of localization in
	wireless networks.
	Attacking a passive Key-less entry system,
	the attacker relays the probing signal emitted from the car to the victims
	customer identification device (CID).
	The CID the emits a signal that opens the doors of the vehicle.
	Now the attacker can enter the car and again relay the signals emitted by the
	car so that he can ignite the engine.
	
	This method has been practically tested on different PKE systems by %TODO \citea{} %DER Artikel
	The authors showed that it is an eqsy and feasible attack.
	It also does not require very specialized hard- or software and
	the requiret components can be easily aquired with a relatively small budget.
	The attack circumvents the cryptography involved in locking and
	operating the car by simple not touching it.

\subsubsection*{Variants}
%two thieves
%three thieves
\subsubsection*{Material \& Methods}
%describe what materials where 

\subsubsection*{Results}
%results francillion et al. achieved with real world cars

\subsection*{Implications}
%what harm can occur from gaining unnotices acces to the vehicle
